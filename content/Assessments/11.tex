\begin{enumerate}
\item 
未定义行为是指程序在约定之外执行时所发生的情况,规范规定了有效的输入是什么以及结果应该是什么。当检测到无效输入是,这也是约定的一部分。没有检测到无效输入,程序继续(错误)。假设输入有效,结果未定义,规范没有说明必须发生什么。

\item 
C++允许未定义行为有两个主要原因。首先,有些操作需要硬件支持,或者在不同的硬件上需要执行不同的操作。在某些硬件系统上得到特定的结果可能非常困难,甚至是不可能的。第二个原因是性能:在所有计算体系结构中保证特定结果的代价非常昂贵。

\item 
不,一个未定义的结果并不意味着这个结果一定是错误的。合理的结果在未定义的行为下也是允许的,只是不能保证而已。此外,未定义行为会污染整个程序。将文件中的相同代码与其他代码一起编译,可能会产生意想不到的结果。新版本的编译器会在未定义行为不会发生的假设下进行更好的优化,所以开发者应该运行杀灭工具,并修复工具报告出来的错误。

\item
出于同样的原因,C++标准保证了性能。如果不增加“正常”情况的开销,就很难正确处理特殊情况,所以可以选择根本不处理特殊情况。更可取的方式是在运行时检测这种情况,这种检测可能也很昂贵。在这种情况下,验证输入应该是可选的。用户提供了无效输入,但是没有运行检测工具,从而程序的行为没有定义,因为算法本身假设输入是有效的,从而导致违背了这个假设。

\end{enumerate}