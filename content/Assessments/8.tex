\begin{enumerate}
\item 
任何为线程安全而设计的数据结构,都必须有事务接口。如果没有标准对存在线程的C++程序提供一些保证,就不可能编写任何可移植的并发C++程序。早在C++11之前就在C++中使用了并发性,但这是由遵循额外标准(如POSIX)的编译器作者实现的。这种情况的缺点是扩展标准各有不同,在没有条件编译和针对每个平台的特定于操作系统的扩展的情况下,没有可移植的方法来编写Linux和Windows的并发程序。类似地,原子操作实现为特定于CPU的扩展。此外,不同编译器所遵循的标准之间也有一些差异,这有时会导致非常难以发现的错误。

\item 
并行算法的使用非常简单,具有并行版本的算法都可以以执行策略作为第一个参数来调用。若是并行执行策略,则算法将在多个线程上运行。另一方面,为了达到最佳性能,可能需要重新设计程序的某些部分。特别是,当数据序列太短(什么是短取决于算法和操作数据元素的成本),并行算法就有优势了。因此,可能有必要重新设计程序,从而操作更大的序列。

\item 
协程是可以暂停正在执行的函数。挂起后,控制权返回给调用者(若这不是第一次挂起,则返回给断点)。协程可以从代码中的任何位置恢复,从不同的函数或另一个协程,甚至从另一个线程恢复。

\end{enumerate}