\begin{enumerate}
\item 
在许多领域中,问题的规模的增长速度与可用的计算资源一样快,甚至更快。随着计算变得越来越普遍,沉重的工作负载可能需要在功率有限的处理器上执行。

\item 
大约在15年前,单核处理能力基本上停止了增长,处理器设计和制造的进步很大程度上转化为更多的处理核和大量的专用计算单元。这些资源不能自动使用,所以需要理解其工作方式。

\item 
效率是指在更多的时间内使用更多的可用计算资源,并且不做任何不必要的工作。性能指的是满足特定的指标,而这些指标取决于计划要解决的问题。

\item
不同的环境中,性能的定义可能完全不同:在超级计算机中,计算的原始速度可能是最重要的,但在交互系统中,只要系统比与它交互的人更快,处理速度可能就没那么重要。

\item
性能必须加以衡量,定量测试结果和分析是对成功和失败都很重要。
	
\end{enumerate}