\begin{enumerate}
\item 
最重要的约束是,程序的结果(或者更严格地说,可观察到的行为)必须不变。这里的门槛很高,只有当能够证明结果对所有可能的输入都是正确的,编译器才允许进行优化。其次考虑的是实用性,编译器必须在编译时间和优化代码的效率之间做出权衡。在启用了最高优化的情况下,证明某些代码转换不会破坏程序的代价可能会很大。 

\item 
除了明显的效果(消除函数调用)外,内联使编译器能够分析更大的代码段。如果没有内联,编译器通常会假设在函数体中“任何事情都是可能的”,编译器可以看到对函数的调用是否产生任何可观察的行为,比如I/O。内联只在一定程度上是有益的。当内联过度时,会增加机器码的大小。此外,编译器很难分析非常长的代码段(片段越长,优化器处理所需的内存和时间就越多)。编译器具有判断某个函数是否值得内联的方法。

\item 
编译器不会进行优化,这通常是因为不能保证这个优化是正确的。编译器对如何使用程序的了解与开发者不同,编译器会假定输入的组合是有效的。另一个常见的原因是,预期的优化不会都那么有效。在这一点上,编译器可能是正确的,若结果表明开发者是正确的,那可以在源代码中进行强制优化。

\item
内联的主要好处不是消除了函数调用的成本,它允许编译器看到函数内部发生了什么,允许对函数调用前后的代码进行分析。将一个更大的代码片段优化为单个基本块时,孤立地考虑每一段代码时,可以将不可能实现的优化变为可能。

\end{enumerate}