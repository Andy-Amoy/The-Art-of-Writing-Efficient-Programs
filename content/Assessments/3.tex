\begin{enumerate}
\item 
现代的CPU有多个计算单元,其中许多可以同时运行。尽可能多地使用CPU计算能力,是使程序效率最大化的方法。

\item 
任何两个可以在同一时间完成的计算,只需要两个计算中较长的时间,另一个看上去不花费任何时间。许多程序中,可以用现成的计算来代替一些将来要完成的计算。这种权衡通常是现在比以后做更多的计算,但只要计算不需要额外的时间,就可以提高整体性能,因为这些计算是与其他工作是并行的关系。

\item 
这种情况称为数据依赖。解决方法是使用流水线,不依赖于任何未知数据的部分计算,将与程序顺序中在其之前的代码并行执行。

\item
条件分支使计算具有不确定性,这就阻碍了CPU对它们进行流水线处理。CPU试图预测将要执行的代码,以便维护流水线。每当这样的预测失败时,流水必须刷新,所有预测错误的指令结果都将丢弃。

\item
根据CPU的分支预测执行的代码可能需要,进行预测性评估。在投机执行的环境中,不能撤消的操作都不能完全提交:CPU不能覆盖内存,不能做任何I/O操作,不能发出中断,也不能报告错误。CPU有必要的硬件来暂停这些操作,直到预测执行的代码确认为真正的代码,或者不是。后一种情况下,投机执行的所有结果都会丢弃,对可见性没有影响。

\item 
一个分支预测良好的程序,通常只对性能的影响很小。因此,两种主要的解决方案是:重写代码,使条件变得更可预测,或者更改计算方式,使用有条件访问的数据。后者称为无分支计算。
	
\end{enumerate}