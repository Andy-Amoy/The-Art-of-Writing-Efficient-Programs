\begin{enumerate}
\item 
为线程安全而设计的数据结构必须有一个事务接口,每个操作要么不能更改数据结构的状态,要么不能将其从一个定义良好的状态转换为另一个定义良好的状态。

\item 
从总体上观察并发代码的性能,共享变量越多,代码就越慢。复杂数据结构通常需要在并发访问的线程之间共享更多的数据。另外,有一些简单的算法(有些是无等待的)允许对数据结构进行有限的线程安全操作。

\item 
使用有效的锁,保护的数据结构不一定会变慢。通常,访问更快。其涉及到共享多少变量,需要多个原子变量的无锁模式可能比单个锁还慢。还必须考虑访问的位置,若数据结构只能在一两个位置(比如一个队列)访问数据,那么锁就会非常有效。如果每次都必须锁定整个数据结构,那么可以同时多个访问的数据结构,性能很可能会很差。

\item
主要的挑战是,向数据结构添加内存通常是一种具有破坏性的操作,需要重排大部分内部数据。在允许对同一数据结构进行并发操作的情况下,很难做到这一点。这与是否由锁保护的数据结构关系不大(有时当一个线程必须管理内存时,锁持有的时间比通常要长得多,但是由于其他原因,长时间的延迟也会发生,程序必须了解到这一点),无锁数据结构中,若内存影响整个数据结构,则很难管理内存。节点数据结构在一个线程上完成所有的内存管理,并使用发布协议向结构中添加新的节点,但是顺序数据结构可能需要数据重新分配,或者需要复杂的内存管理。这种情况下,应该使用\textit{双重检查锁定}来锁定整个数据结构,同时对其内存进行重组。

\item
A-B-A问题是所有节点数据结构无锁实现的常见问题,这些实现使用内存中的数据位置来判断何时发生了更改。不过,在先前删除的节点的内存中分配新节点时,就会出现问题。当另一个线程观察到相同的内存地址时,假设数据结构没有改变,就会产生潜在的数据竞争。这里有多个解决方案,它们使用各种技术来延迟内存的重新分配,直到在同一地址的重新分配不再是一个问题。

\end{enumerate}