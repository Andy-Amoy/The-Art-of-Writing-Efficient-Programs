\begin{flushright}
\zihao{0} 前言
\end{flushright}

高性能编程的艺术仍在归来图中。我还是编程菜鸟的时候,那时候的开发者必须知道每一个数据位的去向(有时的确要这样——用面板上的开关)。现在,计算机有足够的能力完成日常工作,有些细节就不需要那么在意了。不过,还是有一些领域没有足够的计算能力。其实,大多数开发者可以写出高效的代码,这不是一件坏事:在不受性能限制的前提下,开发者就可以专注于如何使代码更好。

本书首先解释了,为什么越来越多的程序员需要关注性能和效率。这将为整本书定下基调,因为它定义了我们在后续章节中所使用的方法论:关于性能的知识最终必须来自测试,并且每个与性能相关的声明都必须有数据进行支持。

其中,有五个要素决定了程序的性能。

首先,我们深入研究所有性能的底层基础:计算硬件(没有交换机——那些日子已经一去不复返了)。从单个部件——处理器和内存——到多处理器计算系统。在此过程中,我们会了解了内存模型、数据共享的成本,还有无锁编程。

高性能编程的第二个部分是对编程语言的有效使用。正是在这一点上,本书针对C++(其他语言也有自己喜欢的低效率)进行具体的说明。而后的是第三个元素,即是帮助编译器提高程序性能的技能。

第四个组成部分是设计。如果设计没有将性能作为其明确目标之一,那么几乎不可能在之后再添加良好的性能。然而,因为这是一个高层次的概念,它汇集了我们之前所学到的所有知识,所以我们最后才来了解性能设计。

高性能编程的最后也是第五个要素就是你——正在阅读本书的你。你的知识和技能水平将最终决定结果。为了帮助你学习,这本书包括了许多例子,可以用于动手探索和自学。学习是一项终身的运动,切勿在看完本书后停止。

\hspace*{\fill} \\ %插入空行
\noindent\textbf{适读人群}

这本书是为有经验的开发人员和程序员编写的,他们从事对性能至关重要的项目,并希望学习不同的技术来提高代码的性能。属于算法交易、游戏、生物信息学、计算基因组学或计算流体动力学社区的开发者,可以从这本书中学习各种技术,并将其应用到他们的工作领域中。

虽然本书使用的是C++语言,但本书的概念可以很容易地转移或应用到其他编译语言,如C、Java、Rust、Go等。

\hspace*{\fill} \\ %插入空行
\textbf{本书内容}

\textit{第1章,性能和并发性}。讨论了关心程序性能的原因,特别是性能好的应用为什么不会凭空出现。我们了解了为了实现最佳性能,甚至是适当的性能,理解影响性能的不同因素和程序特定行为的原因(无论是快执行还是慢执行)。

\textit{第2章,性能测试}。本章内容都是是关于性能测试的。性能有时并不直观,所有涉及效率的决策,从设计选择到具体优化策略,都应该以可靠的数据为指导。本章描述了不同类型的性能评估方式,解释了它们的不同之处,以及应该在什么时候使用,并了解如何在不同的情况下正确地评估性能。

\textit{第3章,CPU架构、资源和性能}。这里,我们开始研究硬件,以及如何高效地使用,从而达到最佳性能。本章将了解CPU资源和能力,以及使用方法。通常CPU资源没有得到最佳利用的原因,以及如何解决这些问题。

\textit{第4章,内存架构与性能}。现在,开始了解现代内存架构,其内在弱点,以及避免或规避这些弱点的方法。对于许多程序来说,性能完全依赖于开发者是否利用硬件特性来提高内存性能,而本章将带各位了解这样的技能。

\textit{第5章,线程、内存和并发}。我们继续学习内存系统,及其对性能的影响,但现在我们将学习扩展到多核系统和多线程领域。事实证明,内存已经是性能的“关键”,当我们增加了并发时,内存管理可能会成为一个大问题。虽然硬件带来的物理限制无法克服,但大多数程序的性能其实还未达到这些限制,而且对于资深的开发者来说,他们的代码效率还有很大的提高空间。所以,本章为读者们提供了必要的知识和工具。

\textit{第6章,并发性和性能}。了解如何为线程安全的程序开发高性能并发算法和数据结构。一方面,为了充分利用并发性,我们必须从高层次的角度看待问题和解决方案策略:数据组织、工作分区,甚至是解决方案的定义都对程序性能产生重大影响。另一方面,性能受到底层因素的极大影响,比如:缓存中数据的分布,即使是最好的设计也可能因糟糕的实现,达不到预期。

\textit{第7章,用于并发的数据结构}。解释并发程序中数据结构的本质,以及在多线程上下文中使用数据结构,比如:定义熟悉的数据结构(如“堆栈”和“队列”等)。

\textit{第8章,C++中的并发}。介绍了C++17和C++20标准中最近添加的并发特性。虽然,现在谈论使用这些特性获得最佳性能的最佳实践还为时过早,但我们可以描述它们的作用,以及编译器当前的支持情况。

\textit{第9章,高性能C++}。这章,将重点从硬件资源的使用,转移到特定编程语言的应用。虽然,我们学到的所有东西都可以直接应用到任何语言中,但本章将讨论C++的特性。读者将了解C++语言的哪些特性可能会导致性能问题,以及如何避免这些问题。本章还会讨论编译器优化的问题,以及开发者如何帮助编译器生成更高效的代码。

\textit{第10章,C++中的编译器优化}。介绍编译器优化,以及开发者如何帮助编译器生成更高效的代码。

\textit{第11章,未定义的行为和性能}。这里有两个重点。一方面,解释了开发者在试图最大化代码性能时,经常忽略的未定义行为的危险性。另一方面,解释了如何利用未定义的行为来提高性能,以及如何正确地指定和记录这些情况。总的来说,与“任何事情都可能发生”相比,这一章提供了一些常见方式来理解未定义行为的问题。

\textit{第12章,为性能而设计}。回顾本书中所学到的所有与性能相关的因素和特性,并探讨所获得的知识,了解如何在开发新软件系统或重新架构现有软件系统时所做出的设计决策。

\hspace*{\fill} \\ %插入空行
\textbf{编译环境}

除了特定于C++效率的章节,不依赖于任何深奥的C++知识。所有的例子都是用C++编写(但是有关硬件性能、高效数据结构和性能设计的课程适用于任何编程语言)。要看懂这些示例,您至少需要具备中等程度的C++知识。

\begin{table}[H]
	\begin{tabular}{|l|l|}
		\hline
		C++ compiler(GCC, Clang, Visual Studio, and so on)                                                                                                                  & 操作系统                                                             \\ \hline
		LLVM版本高于或等于12.x                                                                                                                  & \begin{tabular}[c]{@{}l@{}}Windows, macOS或Linux\end{tabular}                                                             \\ \hline
		\begin{tabular}[c]{@{}l@{}} Profiler(VTune, Perf, GoogleProf, and so on)\end{tabular} &  \\ \hline
		Benchmark Library(GoogleBench)                                                                                                                                  &                                                                                  \\ \hline
	\end{tabular}
\end{table}

每一章都会提到编译和执行示例所需的软件(如果有的话)。大多数情况下,任何现代C++编译器都可以与示例一起使用,除了第8章(C++中的并发),该章要求使用最新版本来完成\textbf{协程}。

\textbf{如果你正在使用这本书的数字版本,我们建议自己输入代码或通过GitHub库访问代码(链接在下一节中提供)。这样做将帮助您避免与复制和粘贴代码相关的任何潜在错误}

\hspace*{\fill} \\ %插入空行
\textbf{下载示例}

您可以从GitHub网站\url{https://	github.com/PacktPublishing/The-Art-of-Writing-Efficient-Programs}下载本书的示例代码文件。如果代码有更新,它将在现有的GitHub库中更新。

我们还有其他的代码包,还有丰富的书籍和视频目录,都在\url{https://github.com/PacktPublishing/}。去看看吧!

\hspace*{\fill} \\ %插入空行
\textbf{联系方式}

我们欢迎读者的反馈。

\textbf{反馈}:如果你对这本书的任何方面有疑问,需要在你的信息的主题中提到书名,并给我们发邮件到\url{customercare@packtpub.com}。

\textbf{勘误}:尽管我们谨慎地确保内容的准确性,但错误还是会发生。如果您在本书中发现了错误,请向我们报告,我们将不胜感激。请访问\url{www.packtpub.com/support/errata},选择相应书籍,点击勘误表提交表单链接,并输入详细信息。

\textbf{盗版}:如果您在互联网上发现任何形式的非法拷贝,非常感谢您提供地址或网站名称。请通过\url{copyright@packt.com}与我们联系,并提供材料链接。

\textbf{如果对成为书籍作者感兴趣}:如果你对某主题有专长,又想写一本书或为之撰稿,请访问\url{authors.packtpub.com}。

\hspace*{\fill} \\ %插入空行
\textbf{欢迎评论}

请留下评论。当您阅读并使用了本书,为什么不在购买网站上留下评论呢?其他读者可以看到您的评论,并根据您的意见来做出购买决定。我们在Packt可以了解您对我们产品的看法,作者也可以看到您对他们撰写书籍的反馈。谢谢你!

想要了解Packt的更多信息,请访问\url{packt.com}。










