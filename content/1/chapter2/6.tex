这一章中,你可能已经学到了整本书中最重要的一课:如果不参照具体的衡量标准,谈论甚至思考性能是没有意义的。剩下的大部分是动手环节:我们介绍了几种测量性能的方法,从整个程序开始,然后深入到每一行代码。

一个大型的高性能项目中,可以看到本章所学到的每一种工具和方法会多次使用。粗略测量——对整个程序或大部分程序进行基准测试和分析——指向需要进一步研究的代码区域。随后会进行更多轮的基准测试或收集更详细的数据文件。最终,将确定需要优化的代码部分,然后问题就变成了:如何才能更快地完成这项工作?此时,可以使用微型基准测试或小型基准测试来测试正在优化的代码。甚至可能会发现,您对这段代码的理解并不如自己所想的那么透彻,需要对其性能进行更详细的分析。同时,不要忘记可以分析微基准测试!

最终,您将得到性能关键代码的新版本,在小型基准测试中看起来是没问题的。但是,不要做任何假设:现在必须通过测量整个程序的性能来确定所做的优化或增强是否有效。有时,这些测量将确认您对问题的理解,并验证解决方案。其他时候,会发现问题并不像想象的那样。优化本身虽然有益,但并没有对整个程序产生预期的效果(甚至会使事情变得更糟)。当有了一个新的数据点,可以比较新旧解决方案的数据文件,并在比较二者差异中寻找答案。

高性能程序的开发和优化几乎从来不是线性的、循序渐进的过程。相反,它有许多迭代,从高级概述到低级数据分析,然后重复。这个过程中,直觉起着作用:只要确保测试和确认您的期望相符即可。因为涉及到性能时,没有什么是真正的显而易见。

下一章,将看到我们之前遇到的问题的解决方案:删除不必要的代码使程序变慢。为了做到这一点,我们必须了解如何有效地使用CPU以获得最大的性能,下一章将专门来讨论这个问题。