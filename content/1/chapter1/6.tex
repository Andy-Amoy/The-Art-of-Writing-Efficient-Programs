In this introductory chapter, we have discussed why the interest in software performance and efficiency is on the rise despite the rapid advances in the raw computational power of modern computers. Specifically, we have learned why, in order to understand the factors limiting performance and how to overcome them, we need to return to the basic elements of computing and understand how computers and programs work at a low level: understanding the hardware and using it efficiently, understanding concurrency, understanding the C++ language features and the compiler optimizations, and their impact on performance.

This low-level knowledge is necessarily very detailed and specific, but we have a plan for dealing with that: as we learn specific facts about the processors or compilers, we will also learn the process by which we have arrived at these conclusions. Thus, at its deepest level, this book is about learning how to learn.

We have further understood that the notion of performance is meaningless without defining the metrics by which this performance is measured. The need to evaluate the performance against the specific metrics implies that any work on performance is driven by data and measurements. Indeed, the next chapter is dedicated to measuring performance.