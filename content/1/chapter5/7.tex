在本章中,我们学习了C++内存模型,以及它给开发者的保证。需要对多线程通过共享数据进行交互时,所发生的详细情况,进行全面的理解。

在多线程程序中,对内存的非同步和无序访问会导致未定义行为,必须不惜任何代价避免。然而,成本通常是通过性能来支付的。虽然我们看重正确性,但当涉及到内存同步时,很容易为正确性付出过高的代价。我们已经了解了管理并发内存访问的不同方法,以及它们的优缺点。最简单的选择是锁定对共享数据的所有访问。另一方面,最复杂的实现使用原子操作,并尽可能少地限制内存序。

性能的第一条规则在这里是完全有效的:性能必须测量,而不是猜测。这对于并发程序来说尤为重要,在并发程序中,由于种种原因,聪明的优化可能无法产生可测量的结果。另一方面,可以保证的一点是,带锁的简单程序更容易编写,而且更有可能是正确的。

有了数据共享对性能影响的理解,就可以更好地理解测量结果,以及在什么时候尝试优化并发内存访问:受内存序限制影响的代码部分越大,放松这些限制就越有可能提高性能。另外,请记住,有些限制来自硬件本身。

总的来说,这比在前面几章中处理的任何内容都要复杂得多(不奇怪,并发通常很难)。下一章将展示在不放弃性能优势的情况下,在程序中如何管理这种复杂性的一些方法。还将看到如何将这里学到的知识,进行实际应用。





