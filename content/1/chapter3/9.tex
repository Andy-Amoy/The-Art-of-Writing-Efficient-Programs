这一章,我们了解了主处理器的计算能力,以及如何有效地使用它们。高性能的关键是最大限度地利用所有可用的计算资源:同时计算两个结果的程序比稍后计算第二个结果的程序快(假设计算能力可用)。我们已经知道,CPU有很多用于各种计算的计算单元,大多数在特定时刻都是空闲的,除非程序得到了高度优化。

我们已经看到,有效使用CPU指令级并行性的主要限制通常是数据依赖:没有足够的工作可以并行完成,以保持CPU繁忙。这个问题的硬件解决方案是流水线操作:CPU不仅在程序的当前点执行代码,而且从未来进行一些计算,这些计算有数据依赖项,也能并行执行它们。只要未来是可知的,这种方法就能很好地工作:如果CPU不能确定这些计算,就不能执行超前的计算。当CPU必须等待决定下一步执行什么机器指令时,流水线就会停止。为了减少这种情况的频率,CPU有特殊的硬件,可以预测最有可能的情况,通过采样条件代码的路径,并有根据的推测地执行代码。因此,程序的性能在很大程度上取决于预测的效果。

我们已经学会了使用特殊工具来帮助测量代码的效率,并识别限制性能的瓶颈。通过测量,我们研究了几种优化技术,这些技术可以使程序利用更多的CPU资源,减少等待时间,增加计算量,并最终提高性能。

本章中,我们略过了每个计算都必须做的事情:访问内存。任何表达式的输入都驻留在内存中,必须在剩下的计算发生之前进入寄存器。中间结果可以存储在寄存器中,但最终必须将某些内容写回内存中,否则整个代码不会产生效果。事实证明,内存操作(读和写)对性能有显著影响,在许多程序中,内存操作是阻碍进一步优化的限制因素。下一章将专门研究CPU与内存的交互。







