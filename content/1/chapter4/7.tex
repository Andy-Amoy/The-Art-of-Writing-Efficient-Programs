本章中,我们了解了内存系统是如何工作的:缓慢的工作,CPU会因内存的低性能而性能变低。但是存储器间距也包含了潜在解决方案:可以用多个CPU操作来换取一个内存访问。

了解到内存系统非常复杂,并且有层级,所以内存系统没有纯粹的速度。如果内存使用的情况非常糟糕,这可能会严重影响程序的性能。同样,也可以把内存看作是一个机会而不是负担:从优化内存访问中获得的收益可能非常大,以至于超过了开销。

正如我们所看到的,硬件本身提供了几种工具来提高内存性能。除此之外,我们还必须使用内存高效的数据结构。如果这还不够,还必须选择内存高效的算法来提高性能。通常,所有的性能决策都必须由测试指标指导和支持。

目前为止,我们所有工作和测量都使用了单个CPU。事实上,从介绍的那几页开始,我们几乎没有提到现在的每一台计算机都有多个CPU核,而且经常有多个物理处理器。原因很简单:我们必须学会有效地使用单CPU,然后才能继续讨论更复杂的多CPU问题。下一章中,我们将注意力转向并发,并有效地使用多核和多处理器系统。