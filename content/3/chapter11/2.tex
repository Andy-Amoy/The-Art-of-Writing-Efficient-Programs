未定义行为(UB)的概念有些神秘,对不了解情况的人来说,它是一种未初始化的警告。Usenet团体comp.std.c警告说:“当编译器遇到(一个未定义的结构)时,它可以合法地让恶魔从你的鼻子里飞出来。”发射核导弹和阉割你的猫(即使你没有养猫)等类似言论,都是在类似的背景下提出的。本章的间接目标是揭开UB的面纱:虽然最终目标是解释UB与性能之间的关系,并展示如何利用UB,但只有在能够理性地讨论这一概念时才能做到。

首先,在C++(或其他编程语言)的上下文中,UB是什么?在标准中有一些特定的地方使用了未定义行为或格式不良的程序。标准进一步说,如果行为没有定义,则标准对结果没有任何要求。相应的情况称为UB。例如,如下代码:

\begin{lstlisting}[style=styleCXX]
int f(int k) {
	return k + 10;
}
\end{lstlisting}

标准规定,如果加法导致整数溢出(也就是说,如果\texttt{k}大于\texttt{INT\_MAX-10}),则上述代码的结果未定义。

当提到UB时,讨论倾向于两个极端,我们刚刚看到的第一个。夸张的语言可能是对UB危险的善意警告,但也会成为理性解释的障碍,因为鼻子和猫都不会受到编译器的攻击。编译器最终会根据程序生成一些代码,然后开发者会运行这些代码。它不会赋予计算机任何超能力:这个程序做的事情,开发者都可以自主完成,例如:用汇编程序手工编写一个相同的指令序列。如果开发者都无法执行\textit{发射核导弹}的机器指令,那么编译器也无法做到,不论UB是否存在(当然,如果你正在为导弹发射控制器编程,这就完全不同了)。底线是,当程序的行为未定义时,根据标准,编译器可以生成开发者不期望的代码。

虽然夸大UB的危险没什么意思,但在另一方面,出现了针对UB进行推理的倾向,这也是一种不祥的做法。例如,考虑以下代码:

\begin{lstlisting}[style=styleCXX]
int k = 3;
k = k++ + k;
\end{lstlisting}

虽然C++标准已经逐步收紧了执行这类表达式的规则,但这个特定表达式的结果在C++17中仍没有定义。许多程序员低估了这种情况的危险,他们觉得,“编译器要么先求k++,要么先求k+k”为了解释为什么这是错误的和危险的,我们必须首先对标准上进行分析。

C++标准有三个相关,且经常混淆的行为类别:实现是定义的、未指定的和未定义的。实现必须提供实现定义的行为的确切规范。这不可选,符合标准的实现必须通过语言构造的行为来符合标准的描述。未指定行为与此类似,但实现没有义务记录该行为:标准通常提供一系列可能的结果,而实现可以指定自己的结果,而不指定会是哪一个。最后,对于未定义行为,标准没有对整个程序的行为强加任何要求。仔细考虑这句话的措辞是至关重要的:标准没有说计算表达式\texttt{k++ + k}的几种备选方法中必须有一种发生(这将是未指定的行为,这不是标准所说的)。标准说整个程序都是病态的,并且对其结果没有任何限制(在为鼻子感到恐慌和恐惧之前,请记住,结果会限制为一些可执行代码)。

无论编译器在编译使用有UB代码行的时候做什么,它必须以标准的方式处理代码的其余部分。因此(这个论证是这样的),损坏仅限于特定代码行可能的结果。正如不夸大危险很重要一样,理解这种观点为什么是错误的也很重要。编译器是基于这样的假设进行的:程序定义良好,在这种情况下(且仅在这种情况下)需要生成正确的结果。如果违背了这个假设,就没有什么先入为主的概念。描述这种情况的一种方法是,编译器不需要容忍UB。我们回到第一个例子:

\begin{lstlisting}[style=styleCXX]
int f(int k) {
	return k + 10;
}
\end{lstlisting}

由于程序定义不明,\texttt{k}大到足以引起整数溢出,因此编译器可以假定这种情况永远不会发生。那真的发生了呢?好吧,如果单独编译这个函数(在一个单独的编译单元中),编译器会生成一些代码,为所有\texttt{k <= INT\_MAX-10}生成正确的结果。如果在编译器和链接器中没有整个程序的转换,那么对于更大的\texttt{k},同样的代码可能会执行,结果与硬件在这种情况下所做事情一样。编译器可以插入对\texttt{k}的检查,但也可能不会(但在某些编译器选项中可能会)。

如果函数是更大的编译单元的一部分呢?这就是事情变得有趣的地方了。编译器现在知道\texttt{f()}函数的输入参数是受限制的,那么这些信息就可以用于优化了。例如,参考如下代码:

\hspace*{\fill} \\ %插入空行
\noindent
\textbf{01\_opt.C}
\begin{lstlisting}[style=styleCXX]
int g(int k) {
	if (k > INT_MAX-5) cout << "Large k" << endl;
	return f(k);
}
\end{lstlisting}

如果\texttt{f()}函数的定义对编译器可见,编译器可以推断出打印输出从来没有发生过:如果\texttt{k}足够大,这个程序可以打印,那么整个程序就是病态的,标准不要求它打印任何东西。如果\texttt{k}的值在某个范围内,程序将不会打印任何东西。无论哪种方式,根据标准,什么都不打印都是有效的结果。请注意,仅仅因为编译器目前没有做优化,并不意味着它永远不会做:这种类型的优化在新的编译器中可能会更加激进。

那么第二个例子呢?表达式\texttt{k++ + k}的结果对于\texttt{k}的值都是未定义的。编译器可以用做什么呢?同样,编译器不需要宽恕UB。这个程序能够保持良好定义的唯一方法是永远不执行这一行。编译器可以先这样假设,然后进行反向推理,从而使让使用这段代码的函数永远不会调用,任何必要的条件都必须为true等,到整个程序永远不会执行。

如果认为编译器不会做那些事情,那这里有一个惊喜:

\hspace*{\fill} \\ %插入空行
\noindent
\textbf{02\_infC}
\begin{lstlisting}[style=styleCXX]
int i = 1;
int main() {
	cout << "Before" << endl;
	while (i) {}
	cout << "After" << endl;
}
\end{lstlisting}

这个程序的自然期望是打印\texttt{Before}并永远挂起。当用GCC(版本9,优化O3)编译时,行为与期望一致。当使用Clang(版本13,也是O3)编译时,它打印Before,然后After,然后立即终止,没有任何错误(不会崩溃,只是退出)。这两个结果都是有效的,因为遇到无限循环的程序的结果是未定义的(除非满足某些条件,而这里没有适用的条件)。

前面的例子对于理解为什么有UB具有指导意义。下一节中,我们将揭开面纱并解释UB产生的原因。

























