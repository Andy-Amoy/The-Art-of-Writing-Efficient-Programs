
设计往往是妥协的艺术,有一些相互竞争的目标和要求必须得到平衡。本节中,将具体讨论与性能相关的权衡。在设计高性能系统时,需要做出许多这样的决定。以下是需要注意的问题。

\subsubsubsection{12.6.1\hspace{0.2cm}接口设计}

本章中,我们已经看到了尽可能少地公开实现的好处。但是,我们从中获得的优化自由与非常抽象的接口的成本之间存在着一种矛盾。 

这种矛盾关系需要在优化不同的组件之间进行权衡:不以任何方式限制实现的接口通常会对客户端造成相当严重的限制,例如:让我们重新回顾点的集合。在不限制其实施的情况下,我们能做什么?不允许任何插入操作,除非在末尾(实现可以是一个\texttt{vector},复制集合的一半在性能上是不可接受的)。我们只能追加到末尾,这意味着不能保持有序,例如:不允许随机访问(集合可以存储在列表中)。如果压缩集合,甚至可能无法提供反向迭代器。给实现者留下几乎无限自由的点集合,现在只能使用前向迭代器(流访问)和追加操作。即使是后者也是一种限制,一些压缩方案要求在读取数据之前对数据进行处理,因此集合可以处于只写或只读状态。

我们给出这个示例,并不是为了演示追求与实现无关的API如何导致对客户端的限制。恰恰相反,这是处理大量数据的有效设计。集合通过添加到末尾进行写入,在写入完成之前,数据没有特定的顺序。最终可能包括排序和压缩。要读取集合,需要动态地解压缩它(如果压缩算法同时在几个点上工作,需要一个缓冲区来保存未压缩的数据)。如果集合需要编辑,可以使用我们在第4章中介绍的算法,来实现内存高效的编辑或字符串,总是从头到尾读取整个集合,并根据需要修改每个点,添加新的点等。我们将结果写入新的集合,并最终删除原来的集合。这种设计允许非常高效的数据存储,无论是在内存使用(高压缩)方面,还是在高效的内存访问方面(仅缓存友好的顺序访问)。它还要求客户端实现流访问和读-改-写的所有操作。 

如果分析数据的访问模式,并认为可以接受流访问和读-改-写更新,那么可以将其作为设计的一部分。当然,不是特定的压缩方案,而是高级的数据组织:在读取之前,必须完成写入,修改数据的唯一方法是将整个集合复制到一个新集合中,在复制期间根据需要修改内容。 

关于这种权衡的一个观察是,我们不仅可能必须在性能需求与易用性或其他设计考虑之间进行平衡,而且通常还需要决定性能的哪个方面更为重要。通常,应该优先考虑底层组件,它们的体系结构对整体设计来说比高级组件的算法选择更重要。因此,以后更改会更加困难,这使得做出明智的设计决策变得更加重要。请注意,在设计组件时,还需要进行必要的权衡。 

\subsubsubsection{12.6.2\hspace{0.2cm}组件设计}

我们已经看到,有时候一个组件要想在设计上有很好的性能,就必须对其他组件进行限制,而其他组件的性能则需要谨慎地选择算法和成熟地实现。但这不是我们必须做出的唯一一个权衡。 

在性能设计中,最常见的平衡行为是为组件和模块选择适当的粒度。制作小组件通常是一种很好的设计实践,特别是在测试驱动的设计中(通常在以可测试性为目标的设计中进行)。另一方面,将系统分割成分割成许多块,并且这些块之间的交互受到限制,则会对性能不利。通常,将较大的数据和代码单元视为单个组件可以实现的更高效。同样,我们的点集合就是一个例子:如果允许无限制地访问集合内的点对象,那么效率会更高。 

最后,这些决定应该通过考虑相互冲突的要求和机会来解决矛盾。最好将一个点作为一个独立的单元,并在其他代码中可测试和重用。但是,我们真的需要将点集合公开为这些点单元的集合吗?或许,可以将它视为包含所存储点信息的集合,而创建点对象只是为了将点读写到集合中,每次一个点。这种方法允许我们保持良好的模块化,并实现高性能。通常,接口是根据清晰且可测试的组件实现的,而在内部,较大的组件以完全不同的格式存储数据。 

应该避免的是在接口中创建“后门”,这些“后门”是专门为绕过遵循良好设计实践而设计的,但现在出现了性能限制。这通常以一种特殊的方式折衷了两个相互竞争的设计目标。相反,最好的方式是重新设计涉及的组件。如果没有找到解决相互矛盾的需求的方法,那么将较小的单元放入内部的、特定于实现的子组件中是个不错的方法。

到目前为止,我们对设计的另一个方面\textit{错误处理}不是很关注,因此这里不会说的太多。

\subsubsubsection{12.6.3\hspace{0.2cm}错误和未定义行为}

错误处理经常当作事后考虑的事情之一,但在设计决策时也应该是挺重要的因素。特别是,对于一个在设计时没有考虑到特定异常处理方法的程序,要将异常安全性(以及扩展为错误安全性)添加进去是非常困难的。

错误处理从接口开始:所有接口本质上都是控制组件之间交互的约定。这些约定应该包括对输入数据的制:如果满足某些外部条件,组件将按照指定的方式运行。但是约定还应该指定如果条件不满足,组件则不能履行约定(或者开发者认为这样做不合适或太困难)会发生什么。 

这个错误响应的大部分也应该包含在约定中:如果指定的需求没有得到满足,组件将以某种方式报告错误。这可以是异常、错误代码、状态标志或其他方法或组合。还有一些书是关于错误处理的最佳实践的,不过本书关注的是性能。

从性能的角度来看,在更常见的情况下,当输入和结果是正确的,并且没有发生任何糟糕的事情时,最重要的考虑因素通常是处理潜在错误的开销。它通常简单地表达为“错误处理必须廉价”。

这意味着,在正常的、无错误的情况下,错误处理必须是廉价的。相反,当这种罕见的事件发生时,我们通常不关心处理错误的代价。这需要根据具体内容因设计而异。 

例如,在处理事务的应用程序中,我们通常需要提交或回滚语义:每个事务要么成功,要么什么都不做。然而,这种设计的性能成本可能很高。通常,失败的事务仍会影响一些更改是可以接受的,只要这些更改不改变系统的主要不变量。对于基于磁盘的数据库,浪费一些磁盘空间也是可以接受的。然后,我们可以为事务分配空间并写入磁盘,但在发生错误的情况下,可以让用户不可访问写入的区域。

在这种情况下,为了提高性能而“隐藏”错误的全部后果,最好设计一个单独的机制来清除这些错误的后果。对于我们的数据库,这样的清理可以在一个单独的低优先级的后台进程中进行,以避免干扰主要访问线程。因为,这是一个通过及时分离来解决矛盾的例子:若必须从错误中恢复,那这么做代价太大,那么代价大的部分可以往后放一放再进行处理。

最后,我们必须考虑这种可能性。即在某些情况下,即使是发现违反约定的行为,但代价太大,第11章就出现过这样的场景。接口约定应明确说明,如果违反了某些限制,则结果是不确定的。如果选择这种方法,不要让程序花时间使未定义的结果更“可接受”。未定义意味任何事情都有可能发生。这不应该轻易地完成,应该考虑替代方法,例如轻量级数据的收集,将耗时的工作留给发生真正错误时处理的代码。但是,明确约定的边界和不确定的结果要比“我们将尽最大努力,但没有承诺”这样不确定的方案要好。 

在设计阶段需要做出许多的权衡,本章并不意味着是一个完整的权衡列表或实现全面平衡的指南。相反,我们展示了几个常见的矛盾和解决它们的方法。 

为了在平衡性能设计目标与其他目标和动机时做出明智的决定,有一些性能估计就很重要了。但我们如何在设计初期就获得性能指标呢?这是我们符性能设计讨论的最后一部分,在某种程度上也是最难的一部分。














