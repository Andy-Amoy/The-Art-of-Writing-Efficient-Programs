
有许多书籍和文章介绍了API设计的最佳实践,通常关注可用性、清晰度和灵活性。常见的指导方针,如“使接口清晰且易于正确使用”和“使接口难以误用”,并不直接处理性能问题,但也不会干扰促进良好性能和效率的实践。前一节中,我们讨论了在为性能设计接口时应该记住的两条重要准则。本节中,我们将探讨一些针对性能的更具体的指导方针。许多高性能程序依赖于并发执行,因此首先解决并发设计是有意义的。

\subsubsubsection{12.4.1\hspace{0.2cm}并发的API设计}

设计并发组件及其接口时,最重要的规则是提供明确的线程安全的保证。注意,“明确”并不意味着“健壮”。事实上,为了获得最佳性能,在底层接口上提供较弱的保证通常性能更好。STL选择的方法是一个很好的例子,所有可能改变对象状态的方法都提供了一个弱保证:只要有一个线程在使用容器,程序就是定义良好的。 

如果想要更强的保证,可以在应用程序级别使用锁。一个更好的方式是创建自己的锁,为接口提供强有力的保证。有时,这些类只是锁的装饰器,将装饰对象的每个成员函数封装在锁中。常见的是,一个锁必须保护多个操作。

为什么?因为在操作完成“一半”之后,让客户端看到特定的数据结构是没有意义的。这将我们引向一个更普遍的情况:作为一条规则,线程安全的接口也应该是事务性的(原子的)。组件(类、服务器、数据库等)的状态在进行API调用之前和调用之后都应该有效,接口保证了所有不变量都应该得到维护。很有可能的情况是,在执行请求的成员函数(用于类)期间,对象经历了一个或多个无效的状态,其不维护指定的不变量。该接口应该使其他线程不能观察到处于这种无效状态的对象。让我们用一个例子来说明。

回想一下上一节的索引树。如果想让这棵树是线程安全的(这是提供强保证的简称),应该让插入新元素是安全的,即使在多个线程同时调用时也是安全的:

\begin{lstlisting}[style=styleCXX]
template <typename T> class index_tree {
	public:
	void insert(const T& t) {
		std::lock_guard guard(m_);
		data_.push_back(t);
		idx_.insert(&(data_[data_.size() - 1]));
	}
	private:
	std::set<T*, compare_ptr<T>> idx_;
	std::vector<T> data_;
	std::mutex m_;
};

\end{lstlisting}

当然,其他方法也必须受到保护。显然,我们不想分别锁定\texttt{push\_back()}和\texttt{insert()}。客户端将如何处理一个对象,客户如何处理数据存储中有新元素但不在索引中的对象?接口甚至没有定义这个新元素是否在容器中,使用迭代器遍历索引也不行。但若使用\texttt{find()}查找数据存储,那么应该是可以的。这种不一致性说明,索引树容器的不变量是在插入前和后维护的。因此,其他线程不能看到这样一个定义不明确的状态,这一点非常重要。我们通过确保接口同时是线程安全的和事务性的来实现这一点。同时调用多个成员函数是安全的,一些线程会阻塞并等待其他线程完成。每个成员函数将对象从一个定义良好的状态移动到另一个定义良好的状态(换句话说,它执行一个事务,比如添加一个新元素)。这两个因素的组合使得对象的使用是安全的。

如果需要一个反例(设计并发接口时不要做什么),回想一下第7章中关于\texttt{std::queue}的讨论。从队列中删除元素的接口不是事务性的:\texttt{front()}返回前端元素但不删除它,而\texttt{pop()}删除前端元素但不返回任何东西。如果队列为空,这两种方法都会产生未定义行为。单独锁定这些方法对我们没有好处,所以线程安全的API必须使用我们在第7章(并发的数据结构)中考虑过的,构造事务性操作的方法,并用锁来保护。

现在转向效率,如果作为容器构建块的各个对象都进行自我锁定,那么效率肯定很低。想象一下,若\texttt{std::deque<T>::push\_back()}本身用一个锁保护,将使\texttt{deque}线程安全(当然,假设其他相关方法也进行锁定)。但这对我们没有好处,因为仍然需要用锁来保护整个事务。它所做的只是浪费一些时间来获取和释放一个我们不需要的锁。

另外,请记住并不是所有的数据都是并发访问的。在将共享状态的数量最小化的程序中,大多数工作都是在特定于线程的数据(对象和独占一个线程的其他数据)上完成的,对共享数据的更新相对较少。独占线程的对象不应该使用锁或其他同步操作。

现在似乎有了一个矛盾:一方面,应该用线程安全的事务接口来设计类和其他组件。另一方面,因为可能正在构建进行锁操作的高级组件,所以不应该用锁或其他同步机制来增加接口的负担。 

解决这一矛盾的方法一般是两种都做:提供非锁定接口(可以用作高级组件的构建块)和提供线程安全接口(有意义的地方)。通常,后者是通过使用锁保护来装饰非锁定接口来实现的。当然,必须在合理范围内进行。首先,非事务性接口都是专为单线程或构建高级接口而存在的,不需要锁定。其次,在特定的设计中,有一些组件和接口是在上下文中使用,也许数据结构是专门为每个线程单独完成的工作而设计的。同样,也没有理由增加并发性的开销。根据设计,有些组件可能仅用于并发,并且是顶级组件——应该具有线程安全的事务接口。这仍然保留了许多类和其他组件,可能以两种方式使用,并且需要锁定和非锁定不同的版本。

根本上说,有两种方法可以做到这一点。第一种是设计一个可以在请求时使用锁的组件,例如:

\begin{lstlisting}[style=styleCXX]
template <typename T> class index_tree {
	public:
	explicit index_tree(bool lock) : lock_(lock) {}
	void insert(const T& t) {
		optional_lock_guard guard(lock_ ? &m_ : nullptr);
		…
	}
	private:
	…
	std::mutex m_;
	const bool lock_;
};
\end{lstlisting}

为此,需要一个条件\texttt{lock\_guard}。可以构建一个使用\texttt{std::optional}或\texttt{std::unique\_ptr},但这种方式不好看,效率还低。编写类似\texttt{std::lock\_guard}的RAII类就要容易得多:

\begin{lstlisting}[style=styleCXX]
template <typename L> class optional_lock_guard {
	L* lock_;
	public:
	explicit optional_lock_guard(L* lock) : lock_(lock) {
		if (lock_) lock_->lock();
	}
	~optional_lock_guard() {
		if (lock_) lock_->unlock();
	}
	optional_lock_guard(const optional_lock_guard&) = delete;
	// Handle other copy/move operations.
};
\end{lstlisting}

除了不可复制之外,\texttt{std::lock\_guard}还是不可移动的。可以遵循相同的设计或使类可移动。对于类,通常可以在编译时而不是运行时处理锁定条件。这种方法使用具有锁定策略的设计:

\begin{lstlisting}[style=styleCXX]
template <typename T, typename LP> class index_tree : private 
LP {
	public:
	void insert(const T& t) {
		std::lock_guard<LP> guard(*this);
		…
	}
};
\end{lstlisting}

至少应该有两个版本的锁定策略LP:

\begin{lstlisting}[style=styleCXX]
struct locking_policy {
	std::mutex m_;
	void lock() { m_.lock(); }
	void unlock() { m_.unlock(); }
};
struct non_locking_policy {
	void lock() {}
	void unlock() {}
};
\end{lstlisting}

现在可以创建具有弱或强线程安全保护的\texttt{index\_tree}对象:

\begin{lstlisting}[style=styleCXX]
index_tree<int, locking_policy> strong_ts_tree;
index_tree<int, non_locking_policy> weak_ts_tree;
\end{lstlisting}

当然,这种编译时方法适用于类,但可能不适用于其他类型的组件和接口。例如,在与远程服务器通信时,可能希望在运行时了解当前会话是共享的,还是独占的。

第二个选项在前面讨论过,锁的装饰器。在这个版本中,原来的类(\texttt{index\_tree})只提供了弱的线程安全保证。这个封装类提供了强保证:

\begin{lstlisting}[style=styleCXX]
template <typename T> class index_tree_ts :
private index_tree<T> 
{
	public:
	using index_tree<T>::index_tree;
	void insert(const T& t) {
		std::lock_guard guard(m_);
		index_tree<T>::insert(t);
	}
	private:
	std::mutex m_;
};
\end{lstlisting}

注意,虽然封装通常优于继承,但这里继承的优点是可以避免复制修饰类的所有构造函数。 

同样的方法也可以应用于其他API,使用显式的参数来控制锁和装饰器。使用哪一种主要取决于设计细节——它们有各自的优点和缺点。注意,即使锁的开销与特定API调用所做的工作相比微不足道,也要避免使用不必要的锁。特别是,这种锁添加了检查死锁的代码。

所有线程安全的接口都应该是事务性的准则,与设计异常安全(或更普遍地说,错误安全的接口)的最佳实践之间有很多重叠。后者更复杂,因为不仅要保证调用接口前后的有效状态,而且在检测到错误后,系统仍要保持良好定义的状态。 

从性能的角度来看,错误处理本质上是开销,并不期望错误会频繁出现(否则,它们就不是真正的错误,而是必须处理的情况)。幸运的是,编写错误安全代码的最佳实践(比如使用RAII对象进行清理)也非常有效,很少会带来明显的开销。尽管如此,一些错误条件是很难检测出来的,正如在第11章中看到的那样。

本节中,了解了一些设计高效并发API的指导原则:

\begin{itemize}
\item 
用于并发使用的接口应该是事务性的。

\item 
接口应该提供最小的必要线程安全保证(对于不打算并发使用的接口提供弱保证)。

\item 
对于既用作客户端可见API,又用作创建自己的、更复杂的事务,需要提供锁的高级组件构建块接口,通常有两个版本:一个具有强线程安全保证,另一个具有弱(或锁定和非锁定)线程安全保证。这可以通过条件锁定或使用装饰器来实现。
\end{itemize}

这些指导原则与设计健壮且清晰的API的其他最佳实践大体一致。因此,为了获得更好的性能,很少需要做出设计上的权衡。 

现在让我们将并发问题抛在脑后,转而讨论性能设计的其他领域。

\subsubsubsection{12.4.2\hspace{0.2cm}复制和发送数据}

这个讨论将在第9章中讨论的问题进行总结,当时讨论了不必要的复制,通常复制都需要发送或接收一些数据。这是一个非常笼统的概念,除了同样普遍的“注意数据传输的成本”外,无法提供任何普遍适用的指导方针。对于一些常见的接口类型,可以对此进行一些说明。 

已经讨论了C++中复制内存的开销,以及对接口的考虑,并在第9章中讨论了实现。对于设计可以强调通用的指导原则,拥有定义良好的数据所有权和生命周期管理。它出现在性能上下文中的原因是过度复制通常是所有权混乱导致的,这是一种解决数据在使用时消失的方法,因为复杂系统的许多部分的生命周期不好理解。 

在分布式程序、客户端-服务器应用程序,或者在带宽限制很重要的组件之间的接口中,都需要管理一些问题。在这样的情况下,通常会使用数据压缩。用CPU时间换取带宽,因为压缩和解压缩数据会消耗处理时间,但传输速度会更快。是否压缩特定通道中的数据不能在设计时做出决定,因为已知信息不够,无法做出权衡。因此,在设计系统时考虑压缩的可能性就很重要,这对于设计可转换为压缩格式的数据结构的接口很重要。如果设计要求压缩整个数据集将其传输,然后将其转换回已解压缩的格式,那么用于处理数据的接口不会改变,但内存需求会增加,因为在某个时刻内存中存储了已压缩和未压缩的状态。另一种是在内部存储压缩数据的结构,在设计接口时需要事先进行考虑。 

举个例子,假设使用简单的结构来存储三维位置和属性:

\begin{lstlisting}[style=styleCXX]
struct point {
	double x, y, z;
	int color;
	… maybe more data …
};
\end{lstlisting}

流行的准则是应该避免getter和setter方法,只访问相应的数据成员。这里,不建议这样做:

\begin{lstlisting}[style=styleCXX]
class point {
	double x, y, z;
	int color;
	public:
	double get_x() const { return x; }
	void set_x(double x_in) { x = x_in; } // Same for y etc
};
\end{lstlisting}

将这些对象存储在一个\texttt{point}集合中:

\begin{lstlisting}[style=styleCXX]
class point_collection {
	point& operator[](size_t i);
};
\end{lstlisting}

这种设计在一段时间内还不错,但随着需求不断变化,现在必须存储和传输数百万个点。很难想象如何用这个接口引入内部压缩,索引操作符返回一个对对象的引用,该对象必须有三个可直接访问的\texttt{double}数据成员。如果有getter和setter,可能已经能够将这个点实现为集合内的一个压缩点集的代理了:

\begin{lstlisting}[style=styleCXX]
class point {
	point_collection& coll_;
	size_t point_id_;
	public:
	double get_x() const { return coll_[point_id_]; }
	…
};
\end{lstlisting}

集合存储压缩数据,并可以动态地解压其中的部分数据,可以通过\texttt{point\_id\_}访问相应的点。

当然,更有利于压缩的接口会要求我们顺序地遍历整个集合。我们刚刚重新回顾了指导方针,该指导方针要求尽可能少地透露关于集合内部工作的信息。对压缩的关注提供了一个特殊的点。如果考虑数据压缩的可能性,或者存储和传输的替代数据的表示,还必须考虑限制对该数据的访问。也许可以想出一种算法,不用随机访问数据就能完成所有需要的计算?如果通过设计限制访问,那么就能保留了压缩数据的可能性(或以其他方式利用有限的访问模式)。

当然,还有其他类型的接口,它们都有与传输大量数据的运行时、内存和存储空间成本相关。进行性能设计时,考虑到这些成本可能对性能至关重要,并尝试限制接口,以实现内部数据表示的最大自由化。当然,都应该在合理的范围内进行,比如手写的配置文件不太可能成为性能瓶颈(无论哪种格式,通常读要比写快)。 

我们已经谈到了数据布局的问题,因为它会影响接口设计。现在让我们直接关注数据组织对性能的影响。


























