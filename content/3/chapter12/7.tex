不仅仅是在做出权衡的决定时,我们必须站在性能数据的基础上。毕竟,如果我们不知道以缓存最优顺序和随机顺序访问数据要花费多少成本时,我们要如何才能决定以实现高效的内存访问呢?这又回到了性能的第一条规则,现在应该已经记住了:永远不要猜测性能。如果程序是散布在白板上的设计图,那么这就是说起来容易做起来难了。 

无法运行设计,那么如何获得测试数据来指导和支持设计决策呢?有些知识来自于经验。并不是指那种“我们一直都是这么做的”的体验。但是,开发者可能已经设计和实现了类似的组件和新系统的其他部分。如果可以重用,则会提供可靠的性能信息。但是,即使必须修改它们或设计类似的东西,也已经有了高度相关的性能指标数据,并且可以很好地转移到新设计中。 

那么,如果我们没有可以用来衡量绩效的相关程序,是否应该这样做呢?这是我们需要依靠模型和原型的时候。模型需要人工构造,据我们所知,它可以模拟未来程序某些部分的预期工作的负载和性能。例如,如果必须决定在内存中组织大量的数据,我们知道经常处理整个数据语料库,微基准,这是一种类型的模型。也可能会使用,为链表和数组处理相同体积的数据组织。这是一个模型,不是对未来项目性能的精确测试,但它提供了有价值的信息,并为决策提供了良好的数据支持。请记住,模型越接近,预测就越不准确:如果对两个可选设计进行建模,并且得出的性能测量值相差不超过10\%时,可能会认为这是完全正确的。顺便说一下,这并不是浪费时间:这样做会获得了重要的信息,两种设计选项都提供了类似的性能,所以可以根据其他标准自由选择。 

并非所有的模型都是微基准测试。有时,可以使用现有的程序来建模新的行为。假设您有一个分布式程序,它对某些数据的操作与下一个程序需要处理的数据类似。新程序将有更多的数据,而且相似性只是表面上的(可能两个程序都处理字符串),因此旧程序不能用于处理新数据的实际测试。没关系,我们可以修改代码来发送和接收更长的字符串。如果现有的程序不使用它们怎么办?这也没关系,我们将编写一些代码以一种比较实际的方式生成和使用这些字符串,并将其嵌入到程序中。现在我们可以启动程序中进行分布式计算的部分,看看发送和接收预期的数据量需要多长时间(假设它需要足够长的时间来压缩)。不过,我们可以做得更好:向代码中添加压缩,并比较网络传输速度与压缩和解压缩的成本。如果不想花费大量时间为特定数据编写实际的压缩算法,那么可以尝试使用现有的压缩库。在免费的库中比较几种压缩算法可以提供更有价值的数据,以便在以后根据数据量决定使用哪个压缩库。 

请仔细注意我们刚才所做的。使用一个现有程序作为框架来运行一些新代码,这些新代码近似于未来程序的行为。换句话说,我们已经构建了一个原型。这是另一种获得原型设计性能评估的方法。当然,为性能构建原型与基于功能的原型有所不同。在后一种情况下,我们希望快速地组合一个系统来演示所需的行为,通常不考虑实现的性能或质量。性能原型应该给我们提供合理的性能数字,因此底层实现必须高效。我们可以忽略特殊情况和错误处理。我们也可以跳过许多特性,只要原型能够执行想要进行基准测试代码即可。有时,我们的原型根本没有任何功能:相反,在代码的某个地方,我们将硬编码一个条件。在实际系统中,在执行某些功能时,会发生这种情况。在这样的原型创建过程中,我们创建的高性能代码通常会在之后形成底层库的基础。

应该指出的是,所有的模型都是近似的,而且即使有一个完整的、最终的代码实现,它们仍然是近似的。微基准测试通常不如大型框架准确,这就产生了像“微基准测试是谎言”这样吸引眼球的标题。微基准测试和其他性能模型,并不总是与最终结果匹配的主要原因是,任何程序的性能都受其环境的影响。例如,可能会为最佳内存访问对一段代码进行基准测试,结果却发现它通常与其他完全饱和内存总线的线程一起运行。 

正如理解模型的局限性很重要一样,不要反应过度也很重要。基准测试确实提供了有用的信息。测量软件越完整、越真实,测量结果越准确。如果基准测试显示一段代码比另一段快几倍,那么当代码在最终上下文中运行,这种差异完全消失的可能性就很低。除了运行在真实数据上的代码的最终版本之外,尝试从其他任何地方获得最后5\%的效率就是很愚蠢的想法了。

原型——模拟真实程序以某种方式再现我们感兴趣的特性的方法——允许我们从不同的设计决策中获得合理的性能评估。它们可以是微观基准册测试,也可以是大型的、已经存在的项目的实验,但它们都有一个目标:将性能设计从猜测的领域,转移到基于测量驱动的决策上。 
