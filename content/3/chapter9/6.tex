本章中,我们从语言的角度讨论了C++效率的两个主要方面中的第一个:避免低效的语言结构,这可以归结为只做必要的工作。我们研究过的许多优化技术都与前面的内容相吻合,比如:访问内存的效率和避免并发程序中的错误共享。

每个开发者面临的一个大难题是,在编写高效代码之前应该投入多少工作,以及应该留给优化的工作多少时间。首先,高性能始于设计阶段:设计不会锁定在低性能和低效实现中的体系结构和接口,是开发高性能软件的重中之重。 

除此之外,还应该区分过早的优化和不必要的悲观。创建临时变量以避免混叠是不明智的,除非性能测试数据显示,正在优化的函数对总体执行时间有很大的贡献(或者提高了可读性,这是另一回事)。在分析器告诉之前,按值传递大型数据只会使运行效率下降,所以应该从一开始就避免这样做。 

两者之间的界限并不明确,所以必须权衡几个因素。需要考虑修改对程序的影响:会使代码更难阅读、更复杂,还是更难测试?通常,不希望为了性能而冒险制造更多的Bug,除非测试报告说必须这样做。另一方面,有时可读性更强或更简单的代码也是更高效的代码,因此不能认为优化是过早的。 

C++效率的第二个方面是帮助编译器生成更高效的代码。我们将在下一章讨论这个问题。