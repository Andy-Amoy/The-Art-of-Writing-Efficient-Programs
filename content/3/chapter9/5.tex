在经历了不必要的计算和低效的内存使用之后,低效的编码、不充分利用大部分可用计算资源代码的下一个问题,可能就是不能很好地流水线化代码。我们已经在第3章中看到了CPU流水线的重要性,还了解了流水线最糟糕的干扰因素通常是条件操作,特别是硬件分支预测器无法猜测的操作。 

但优化条件代码以实现更好的流水是C++最困难的优化之一。只有当分析器显示出较差的分支预测时,才应该进行此操作。然而,错误预测的分支的数量并不一定要很大才会认为是“差”的:好的程序通常会有少于0.1\%的错误预测分支。1\%的预测误差率是相当大的。如果不检查编译器输出(机器码),也很难预测源代码的优化效果。

如果分析器显示了一个预测不好的条件操作,下一步就是确定哪个条件预测错了。例如:

\begin{lstlisting}[style=styleCXX]
if (a[i] || b[i] || c[i]) { … do something … }
\end{lstlisting}

即使整体结果可预测,也可能产生一个或多个预测不佳的分支。这与C++中布尔逻辑的定义有关。操作符\texttt{||}和\texttt{\&\&}可以短路:表达式从左到右求值,直到结果已知。如果\texttt{a[i]}为true,则代码就不管数组元素\texttt{b[i]}和\texttt{c[i]}了。有时,这是必要的。实现的逻辑可能不存在这些元素,但布尔表达式会无缘无故地引入不必要的分支。前面的\texttt{if()}语句需要3个条件操作。而在下面的代码中:

\begin{lstlisting}[style=styleCXX]
if (a[i] + b[i] + c[i]) { … do something … }
\end{lstlisting}

如果值\texttt{a}、\texttt{b}和\texttt{c}非负,但需要单个条件操作,则与最上一个示例相同。再次强调,这不是那种应该先做的优化,除非有测试确认这里需要优化。

看一下这个函数:

\hspace*{\fill} \\ %插入空行
\noindent
\textbf{05\_branch.C}
\begin{lstlisting}[style=styleCXX]
void f2(bool b, unsigned long x, unsigned long& s) {
	if (b) s += x;
}
\end{lstlisting}

如果\texttt{b}值不可预测,则效率非常低。需要更好的表现,只是一个改变:

\hspace*{\fill} \\ %插入空行
\noindent
\textbf{05\_branch.C}
\begin{lstlisting}[style=styleCXX]
void f2(bool b, unsigned long x, unsigned long& s) {
	s += b*x;
}
\end{lstlisting}

这种改进可以通过一个简单的基准测试来确定:原始的、有条件的、实现与无分支的实现:

\begin{tcblisting}{commandshell={}}
BM_conditional   176.304M items/s
BM_branchless     498.89M items/s
\end{tcblisting}

如您所见,无分支实现比有条件的快3倍。

重要的是,不要在这种类型的优化上走极端。这种优化必须由测试驱动,原因如下:

\begin{itemize}
\item
分支预测器相当复杂,对它们能处理和不能处理的直觉大概率是错误的。

\item
编译器优化常常会改变代码,因此,如果不测量或检查机器码,即使我们知道会对分支进行预测,猜测的结果可能是错的。

\item
即使分支错误地预测,性能影响也会变化,因此在没有测试的情况下没法进行确定。
\end{itemize}

例如,手动优化这段代码几乎没什么用:

\begin{lstlisting}[style=styleCXX]
int f(int x) { return (x > 0) ? x : 0; }
\end{lstlisting}

这看起来像条件代码,若\texttt{x}的符号是随机的,则不可预测。然而,分析器很可能不会在这里显示预测错误的分支,原因是大多数编译器不会使用条件跳转来实现这一行。在x86上,一些编译器将使用CMOVE指令,它执行一个条件移动。根据条件,将值从两个源寄存器之一移动到目标寄存器。这条指令的条件性质是良性的,条件代码的问题是CPU事先不知道接下来执行哪条指令。在有条件的移动实现中,指令序列完全线性,顺序是预先确定的,所以不需要猜测。 

另一个不太可能从无分支优化中受益的例子是条件函数调用:

\begin{lstlisting}[style=styleCXX]
if (condition) f1(… args …) else f2(… args …);
\end{lstlisting}

无分支实现可以使用函数指针数组:

\begin{lstlisting}[style=styleCXX]
using func_ptr = int(*)(… params …);
static const func_ptr f[2] = { &f1, &f2 };
(*f[condition])(… args …);
\end{lstlisting}

如果函数最初是内联的,那么用间接函数调用替换会影响性能。否则,这个更改可能没有任何作用。在编译期间跳转到另一个地址未知的函数,其效果类似于错误地预测分支,所以这段代码会使CPU刷新流水线。 

优化分支预测需要很多技巧。性能结果可能有改进,也可能没改进(或者只是浪费了一些时间),所以每一步都要有性能测试进行指导。

我们现在已经了解了许多C++程序中潜在的低效率情况,以及改进它们的方法。最后,我们会给出了一些优化代码的指南。



