开发者经常谈论一种语言是否有效。特别是C++,它的开发有着明确的效率目标。与此同时,它在某些圈子里认为C++是低效的。这是怎么回事呢?

效率在不同的情况下或对不同的人有不同的含义。例如:

\begin{itemize}
\item
C++的设计遵循\textbf{零开销}原则:除了少数的例外,不会为任何不使用的特性支付运行成本。从这个意义上说,C++可以认为是高效的语言。

\item
必须为所使用的语言特性付出一些代价,若将它们转化为一些运行时的话。C++的优点是不需要任何运行时代码,来完成可以在编译期间完成的工作(尽管编译器的实现和标准库的效率各不相同)。高效的语言不会为执行请求的代码增加任何开销,C++在这里做的还是相当不错的,我们将在下面讨论一个主要的限制。

\item
如果上述情况属实,那么C++是如何被持这种观点的人贴上“效率低下”的标签的呢?现在我们来看另一个视角定义的效率:用这种语言编写高效的代码有多容易?或者,做一件看起来很自然,但实际上是一种非常低效的解决问题的方法有多容易?与之前提到的问题密切相关,我们也在上一段中提到:C++能够非常高效地完成开发者要求它做的事情。但是,要在语言中准确地表达想要的东西并不容易,而且编写代码的方式有时会强制约束和要求开发者,这些约束和要求有些是有运行时成本的。
	
\end{itemize}

从语言设计者的角度来看,最后一个问题不是语言的低效:你要求机器做X和Y,它需要时间来做X和Y,没有做任何超出要求的事情。但从开发者的角度来看,如果开发者只想做X而不关心Y,那么这是一种效率低下的语言。 

本章的目标是帮助读者编写代码,并清楚地表达希望机器做去什么。这样做的目的有两个。若认为你的代码主要读者是编译器:通过精确地描述想要的事情和了解编译器可以自由修改的内容,可以给编译器生成更高效代码的自由。但是对于程序的读者来说也是一样的:他们只能推断出作者在代码中表达的内容,而不能推断出想要表达的内容。如果代码的某些方面发生了变化,那么优化代码是否安全?这种行为是有意为之,还是错误的实现,并且可以去更改吗?这再次提醒我们,编程更多的是一种与同伴交流的方式,而不仅仅是与机器交流的方式。

我们将从显而易见的简单低效代码示例开始。但这种情况,甚至会出现在该语言资深开发者的代码中。














