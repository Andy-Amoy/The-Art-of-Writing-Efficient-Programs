本章中,我们研究了C++效率的第二个主要方面:帮助编译器生成更高效的代码。 

本书的目标是帮助读者理解代码、计算机和编译器之间的相互作用,以便读者们能够判断和理解编译器做出的这些决定。 

帮助编译器优化代码的最简单的方法是遵循有效优化的经验规则,其中许多也是好的设计规则:尽量减少代码不同部分之间的接口和交互,将代码组织成块、函数和模块,每个模块都有简单的逻辑和定义良好的接口边界,避免全局变量和其他隐藏交互等。这些也是最佳设计实践其实并非巧合:通常,程序员容易阅读的代码,编译器会更容易分析。

更高级的优化通常需要检查编译器生成的代码。如果有注意到编译器没有做一些优化,考虑一下是否存在这种优化无效的情况:不要考虑程序中发生了什么,而是考虑给定代码片段中可能发生的事情(例如,你可能知道自己从不使用全局变量,但编译器会假设你可能会使用全局变量)。 

下一章中,我们将探索C++的一个非常微妙的领域(以及一般的软件设计),它可能与性能研究有意想不到的重叠。