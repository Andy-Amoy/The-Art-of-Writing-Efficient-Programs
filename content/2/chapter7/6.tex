本章最重要的是,设计并发数据结构是困难的,应该利用一切机会进行简化。应用程序使用特定的数据结构,根据相应的限制,可以让数据结构更简单和更快。 

首先,必须做出的决定是代码的哪些部分需要线程安全,哪部分不需要。通常,最好的解决方案是给每个线程自己的数据:单个线程使用的自己数据,根本不需要考虑线程安全的问题。当这没办法使用时,寻找其他特定于应用程序的限制:是否有多个线程修改特定的数据结构?如果只有一个写线程,实现通常会更简单。有什么特定于应用程序的保证可以利用吗?知道数据结构的最大尺寸吗?是否需要同时从数据结构中删除数据以及添加数据,或者可以及时分离这些操作?在一些数据结构不变的情况下,是否存在定义良好的时间段?如果是,则不需要任何同步来读取。这些和许多其他特定于应用程序的限制,可以用来提高数据结构的性能。 

第二个重要决策是:将支持数据结构上的哪些操作?重申最后一段的另一种方法是“实现最小化的必要接口”。实现的接口都必须是事务性的:每个操作都必须具有定义良好的数据结构的状态行为。只有在数据结构处于某种状态时才有效的操作,不能在并发程序中安全使用,除非使用方使用客户端,将多个操作锁定组合到一个事务中(在这种情况下,这些组合可能一开始就是一个操作)。

本章还介绍了几种实现不同类型数据结构的方法,以及评估其性能的方法。最终,只有在实际应用的背景下,使用实际数据才能获得准确的性能。然而,有用的近似基准,可以在开发和评估潜在的替代方案时节省大量时间。 

本章结束了我们对并发的探索。接下来,我们继续学习C++语言本身如何影响程序的性能。