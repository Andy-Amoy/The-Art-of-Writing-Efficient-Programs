上一章中,我们了解了影响并发程序性能的基本因素。现在是时候将这些知识用于实际,并学习如何为线程安全的程序开发高性能并发算法和数据结构。

一方面,要充分利用并发性,必须以高阶视角看待问题和解决方案策略:选择数据组织、工作分区,有时甚至是解决方案的定义都是对程序性能产生重大影响。另一方面,性能受到底层因素的极大影响,比如缓存中数据的排列,即使是最好的设计也可能被糟糕的实现所破坏。这些底层的细节常常难以分析,难以用代码表达,并且需要非常仔细的编码。我们不期望这些代码散落的到处都是,因此必须对它们进行封装,所以需要考虑封装这种复杂性的最佳方式。

本章将讨论以下内容:

\begin{itemize}
\item 高效的并发性
\item 锁的使用、锁的缺陷,以及无锁编程的介绍
\item 线程安全的计数器和累加器
\item 线程安全的智能指针
\end{itemize}





