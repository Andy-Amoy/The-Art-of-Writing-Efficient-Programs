从根本上说,使用并发性来提高性能非常简单:实际上只需要做两件事。第一种方法是为并发线程和进程提供足够的工作,使它们始终处于忙碌状态。第二个是减少共享数据的使用,并发访问共享变量的开销非常大。剩下的只是实现的问题。

不幸的是,实现往往相当残酷,而且当期望的性能增益更大,并且当硬件变得更强大时,难度就会增加。这是由于Amdahl法则,这是每个从事并发工作的开发者都听说过的,但并不是每个人都完全理解它的含义。

法则本身很简单:对于一个具有并行(可扩展)部分和单线程部分的程序,最大可能的加速\textit{s}如下所示:

\begin{center}
$ s = \dfrac{s_0}{s_0(1-p)+p} $
\end{center}

这里,计算是程序并行部分的加速比,也是程序并行部分的分数。现在考虑一下在大型多处理器系统上运行程序的情况:如果有256个处理器,并且能够充分利用它们,除了运行时间的1/256,程序的总加速会限制为128,也就是说,加速比被削减了一半。换句话说,如果只有1/256的程序是单线程的,或者是在锁下执行的,那么不管我们如何优化程序的其余部分,在这个256个处理器的系统的加速比永远不会超过50\%。

这就是为什么在开发并发程序时,设计、实现和优化的重点应该是使单线程计算并发化,并减少程序访问共享数据所花费的时间。

第一个目标,使计算并行化,从算法的选择开始,但是许多设计决策会影响结果,所以我们应该更多地了解它。第二种方法是降低数据共享的成本,延续了上一章的主题:当所有线程都在等待访问某个共享变量或锁(它本身也是一个共享变量)时,程序实际上是单线程的,只有当前有访问权限的线程在运行,这就是为什么全局锁和全局共享数据对性能特别不利的原因。但是,即使是多个线程之间共享的数据,如果并发访问这些线程,也会限制这些线程的性能。

正如之前提到的,数据共享的需求基本上是由问题本身导致的。特定问题的数据共享量都可能受到算法、数据结构选择和其他设计决策以及实现的影响。有些数据共享是实现的产物,或数据结构选择的结果,但其他共享数据则是问题本身所固有的。如果需要计算满足某个属性的数据元素,比如只有一个计数,所有线程必须将其更新为共享变量。然而,实际发生了多少共享,以及对总体程序加速的影响,在很大程度上取决于具体实现。

在本章中,我们将追寻两条线索:首先,考虑到一些数据共享不可避免,我们将研究如何使这个过程更有效。然后,考虑设计和实现技术,这些技术可以用来减少数据共享的需求或用于等待访问该数据的时间。那我们从第一个问题开始把,如何进行高效的数据共享。
































