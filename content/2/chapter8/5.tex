C++11是第一个添加线程存在的标准,为并发环境中记录C++程序的行为奠定了基础,并在标准库中提供了一些有用的功能。除了这些功能之外,基本的同步原语和线程也是有用的。后续版本对这些特性进行了改进。

C++17以并行STL的形式带来了重大的进步。当然,性能由实现决定。只要数据足够多,观察到的性能就会更好,即使是在搜索和分区等难以并行化的算法上。然而,如果数据序列太短,并行算法的性能并不是很好。

C++20增加了协程支持,我们已经从理论上和一些基本的示例中了解了无栈协程的工作方式。然而,现在谈论使用C++20协程的性能和最佳实践还为时过早。

本章结束了本书对并发的探索。接下来,我们继续学习C++语言本身的使用如何影响程序的性能。